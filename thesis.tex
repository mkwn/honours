%% LyX 2.0.2 created this file.  For more info, see http://www.lyx.org/.
%% Do not edit unless you really know what you are doing.
\documentclass[11pt,english]{article}
\usepackage[T1]{fontenc}
\usepackage[latin9]{inputenc}
\usepackage{geometry}
\geometry{verbose,tmargin=3cm,bmargin=3cm,lmargin=3cm,rmargin=3cm,headheight=3cm,headsep=3cm,footskip=1.5cm}
\setlength{\parskip}{\bigskipamount}
\setlength{\parindent}{0pt}
\usepackage{amsthm}
\usepackage{amsmath}
\usepackage{amssymb}
\usepackage{setspace}
\usepackage{esint}
\setstretch{1.5}

\makeatletter
%%%%%%%%%%%%%%%%%%%%%%%%%%%%%% Textclass specific LaTeX commands.
\theoremstyle{plain}
\newtheorem{thm}{\protect\theoremname}[section]
  \theoremstyle{plain}
  \newtheorem{cor}[thm]{\protect\corollaryname}
  \theoremstyle{plain}
  \newtheorem{lem}[thm]{\protect\lemmaname}
  \theoremstyle{plain}
  \newtheorem{prop}[thm]{\protect\propositionname}
  \theoremstyle{plain}
  \newtheorem{conjecture}[thm]{\protect\conjecturename}
  \theoremstyle{definition}
  \newtheorem{defn}[thm]{\protect\definitionname}
  \theoremstyle{definition}
  \newtheorem{example}[thm]{\protect\examplename}
  \theoremstyle{remark}
  \newtheorem{rem}[thm]{\protect\remarkname}
  \theoremstyle{remark}
  \newtheorem{claim}[thm]{\protect\claimname}

%%%%%%%%%%%%%%%%%%%%%%%%%%%%%% User specified LaTeX commands.
%clickable links
\usepackage[bookmarks,hidelinks]{hyperref} 

%\ref{thm:asdf} instead of Theorem~\ref{thm:asdf}
\usepackage[nameinlink,capitalise]{cleveref}
\AtBeginDocument{\renewcommand{\ref}[1]{\cref{#1}}}
\theoremstyle{plain}
\newtheorem{mythm}{\protect\theoremname}[section]
\renewenvironment{thm}{\begin{mythm}}{\end{mythm}}
\theoremstyle{definition}
\newtheorem{mydefn}[mythm]{\protect\definitionname}
\renewenvironment{defn}{\begin{mydefn}}{\end{mydefn}}
\theoremstyle{definition}
\newtheorem{myexample}[mythm]{\protect\examplename}
\renewenvironment{example}{\begin{myexample}}{\end{myexample}}
\theoremstyle{plain}
\newtheorem{myprop}[mythm]{\protect\propositionname}
\renewenvironment{prop}{\begin{myprop}}{\end{myprop}}
\theoremstyle{plain}
\newtheorem{mycor}[mythm]{\protect\corollaryname}
\renewenvironment{cor}{\begin{mycor}}{\end{mycor}}
\theoremstyle{plain}
\newtheorem{mylem}[mythm]{\protect\lemmaname}
\renewenvironment{lem}{\begin{mylem}}{\end{mylem}}
\theoremstyle{plain}
\newtheorem{myconjecture}[mythm]{\protect\conjecturename}
\renewenvironment{conjecture}{\begin{myconjecture}}{\end{myconjecture}}
\theoremstyle{remark}
\newtheorem{myrem}[mythm]{\protect\remarkname}
\renewenvironment{rem}{\begin{myrem}}{\end{myrem}}
\theoremstyle{remark}
\newtheorem{myclaim}[mythm]{\protect\claimname}
\renewenvironment{claim}{\begin{myclaim}}{\end{myclaim}}
\crefformat{equation}{#2(#1)#3}

%ordered lists should use parens instead of a point
%\renewcommand\theenumi{\arabic{enumi}}
%\renewcommand\labelenumi{(\theenumi)}

%\left(\right) should behave the same as ()
\let\originalleft\left
\let\originalright\right
\renewcommand{\left}{\mathopen{}\mathclose\bgroup\originalleft}
\renewcommand{\right}{\aftergroup\egroup\originalright}

%table of contents spacing tweaks
\usepackage{tocloft}
\setlength\cftparskip{-2pt}
\setlength\cftbeforesecskip{1pt}
\setlength\cftaftertoctitleskip{2pt}

%comment environment
\usepackage{verbatim}

%mainly for light colours color!percent
\usepackage{xcolor}

%shaded WIP notes
\usepackage{framed}
\usepackage{lipsum}
\colorlet{shadecolor}{blue!20}
\newenvironment{note}
{\colorlet{shadecolor}{blue!20}\begin{shaded}}{\end{shaded}}
\newenvironment{todo}
{\colorlet{shadecolor}{red!20}\begin{shaded}}{\end{shaded}}

\makeatother

\usepackage{babel}
  \providecommand{\claimname}{Claim}
  \providecommand{\conjecturename}{Conjecture}
  \providecommand{\corollaryname}{Corollary}
  \providecommand{\definitionname}{Definition}
  \providecommand{\examplename}{Example}
  \providecommand{\lemmaname}{Lemma}
  \providecommand{\propositionname}{Proposition}
  \providecommand{\remarkname}{Remark}
\providecommand{\theoremname}{Theorem}

\begin{document}

\title{Stein's Method}


\author{Matthew Kwan}

\maketitle
\begin{comment}
\begin{thm}
t\end{thm}
\begin{cor}
c\end{cor}
\begin{lem}
l\end{lem}
\begin{prop}
p\end{prop}
\begin{conjecture}
c\end{conjecture}
\begin{defn}
d\end{defn}
\begin{example}
e\end{example}
\begin{rem}
r\end{rem}
\begin{claim}
c
\end{claim}
\end{comment}

\global\long\def\floor#1{\left\lfloor #1\right\rfloor }


\global\long\def\ceil#1{\left\lceil #1\right\rceil }


\global\long\def\i{i}


\global\long\def\ii{j}


\global\long\def\NN{\mathbb{N}}


\global\long\def\ZZ{\mathbb{Z}}


\global\long\def\RR{\mathbb{R}}


\global\long\def\CC{\mathbb{C}}


\global\long\def\F{\mathcal{F}}


\global\long\def\d{\operatorname d}


\global\long\def\id{\operatorname{id}}


\global\long\def\one{\operatorname{1}}


\begin{note}Commentary in blue\end{note}\begin{todo}Issues in red\end{todo}

\tableofcontents{}


\section{Introduction}

\begin{note}

introduce with limit theorems: Central Limit theorem, Poisson Limit
theorem. Failure of limit theorems: they provide no understanding
of speed of convergence, in particular convergence cannot be assumed
to be uniform as parameters vary.

Stein's method is a technique for bounding the distance between distributions,
with a variety of different distance metrics. Quantitative bounds
can be useful in their own right, or can be further applied to prove
asymptotic results.

\end{note}


\subsection{Notation}

For this thesis, the set of natural numbers $\NN$ includes zero.
We write $\one_{A}$ for the characteristic function of $A$: $\one_{A}\left(x\right)=1$
if $x\in A$, otherwise $\one_{A}\left(x\right)=0$.

Unless otherwise specified, all asympotics are as $n\to\infty$. Apart
from standard asymptotic notation, we use two notions of asymptotic
equivalence: $f\sim g$ means $f=g\left(1+o\left(1\right)\right)$
and $f\asymp g$ means $f=O\left(g\right)$ and $g=O\left(f\right)$.


\part{Theory}


\section{General Probability Theory}

\global\long\def\L#1{\mathcal{L}_{#1}}


\DeclareRobustCommand{\L}[1]{\ifmmode{\mathcal{L}_{#1}}\else\polishL\fi}

\global\long\def\Om{\Omega}


\global\long\def\om{\omega}


\global\long\def\cA{\mathcal{A}}


\global\long\def\B{\mathcal{B}}


\global\long\def\E{\mathbb{E}}


\global\long\def\F{F}


\global\long\def\N{\mathcal{N}}


\global\long\def\Po{\operatorname{Po}}


\global\long\def\Var{\operatorname{Var}}


\global\long\def\im{\operatorname{im}}


\global\long\def\Pr{\mathbb{P}}


\global\long\def\m{\mu}


\global\long\def\X{X}


\global\long\def\bX{\mathbf{X}}


\global\long\def\A{A}


\global\long\def\P{\mathcal{P}}


\begin{comment}

For many combinatorial applications, an informal understanding of
probability theory will suffice, because probability spaces of combinatorial
objects are usually finite. However, we will need a more rigorous
foundation in probability theory. The following is only intended as
a brief review (cite textbook).


\subsection{Measure theory}

Let $\A$ be a collection of subsets of some set $\Om$. We say $\cA$
is a \emph{$\sigma$-algebra} if it is closed under countable unions
and complements, and contains the empty set. A \emph{measure} on $\cA$
is a map $\m:\cA\to\left[0,\infty\right)$ such that for any collection
of pairwise disjoint sets $\left\{ \A_{\i}\right\} _{\i\in\NN}\subseteq\cA$,
we have $\sum_{\i\in\NN}\m\left(A_{\i}\right)=\m\left(\bigcup_{i\in\NN}\A_{\i}\right)$.

A \emph{measure space} consists of a set $\Om$, a $\sigma$-algebra
$\cA$ of subsets of $\Om$ and a measure $\m$ on $\cA$. We represent
a measure space by the triple $\left(\Om,\cA,\m\right)$.
\begin{itemize}
\item define $\sigma$-algebra generated by a set
\item define Borel $\sigma$-algebra $\B$
\item if $\Om$ is countable we can choose $\cA=\Om$
\item define measurable functions $X:\Om_{1}\to\Om_{2}$
\item define integration
\end{itemize}
\end{comment}


\subsection{Review of basic concepts}

\begin{note}

I'm a little bit uncertain how much depth to go into for this. At
the moment, it's written so that someone who's seen measure theory
but no probability theory (an analyst) can understand. Where possible,
I've tried to translate things into the discrete case, because it's
often more intuitive (and since I plan for applications to be combinatorial).

\end{note}

For many combinatorial applications, an informal understanding of
probability theory will suffice. However, in this thesis a rigorous
foundation in probability theory will be useful. The following is
intended only as a brief review.
\begin{defn}
A \emph{probability space} is a measure space $\left(\Om,\cA,\Pr\right)$
with $\Pr\left(\Om\right)=1$. In this case we say $\Pr$ is a \emph{probability
measure}, and denote the set of all probability measures on $\left(\Om,\cA\right)$
by $\P\left(\Om,\cA\right)$ or $\P\left(\Om\right)$ if there is
no ambiguity. An \emph{event} is a measurable set $\A\in\cA$.
\end{defn}

\begin{defn}
A \emph{probability space} is a measure space $\left(\Om,\cA,\Pr\right)$
with $\Pr\left(\Om\right)=1$. In this case we say $\Pr$ is a \emph{probability
measure}, and denote the set of all probability measures on $\left(\Om,\cA\right)$
by $\P\left(\Om,\cA\right)$ or $\P\left(\Om\right)$ if there is
no ambiguity. An \emph{event} is a measurable set $\A\in\cA$.
\end{defn}
For our purposes $\Om$ will often be a finite set of combinatorial
objects, with $\cA$ as the power set of $\Om$. In this case $\Pr$
is defined by $\Pr\left(\om\right):=\Pr\left(\left\{ \om\right\} \right)$,
for each $\om\in\Om$. We will discuss specific probability spaces
on combinatorial objects in \ref{sec:random-structures}, but we include
a particularly useful definition here:
\begin{defn}
In a probability space $\left(\Om,\cA,\Pr\right)$ where $\Om$ is
finite, if $\Pr\left(\om\right)=1/\left|\Om\right|$ for each $\om\in\Om$,
then we say $\Pr$ is \emph{uniform}.
\end{defn}
For an event $\A$, $\Pr\left(\A\right)$ is interpreted as the ``probability
that $\A$ occurs''. For combinatorial spaces, events are usually
of the form $\A=\left\{ \om\in\Om:\, P\left(\om\right)\mbox{ holds}\right\} $,
where $P\left(\om\right)$ is some property of an object $\om$. For
clarity, we often abuse notation slightly and write $\Pr\left(P\left(\om\right)\mbox{ holds}\right)$
instead of $\Pr\left(\A\right)$.
\begin{defn}
A \emph{random element} $\X:\Om_{1}\to\Om_{2}$ is a measurable function
from a probability space $\left(\Om_{1},\cA_{1},\Pr\right)$ to some
measure space $\left(\Om_{2},\cA_{2},\m\right)$. If the target measure
space is $\RR^{n}$ with the Borel $\sigma$-algebra and the Lebesgue
measure, then we say $\X$ is a \emph{random vector}; if $n=1$ then
$\X$ is a \emph{random variable}. If $\X$ only takes countably many
values then we say $\X$ is \emph{discrete}.
\end{defn}
If the underlying probability space $\Om_{1}$ is countable, then
any function is measurable.

We will often be interested in the probability that a random element
takes certain values, without regard to the underlying probability
space.
\begin{defn}
Suppose $\X$ is a random element with target measure space $\left(\Om,\cA,\m\right)$.
The \emph{distribution} (or \emph{law}) $\L{\X}$ of $\X$ is the
pushforward measure with respect to $\X$. That is, it is a probability
measure defined by $\L{\X}\left(\A\right)=\Pr\left(\X^{-1}\left(\A\right)\right)$
for $\A\subseteq\cA$.
\end{defn}
It is worth noting that in fact any probability measure is the distribution
of some random element. To see this, note that given a probability
measure $\Pr\in\P\left(\Om\right)$, we can choose $X=\id_{\Om}$
to have $\L{\X}=\Pr$. So, it is often convenient to specify random
variables by their distributions, without defining an underlying probability
space. We can use slightly abusive (but standard) notation like $\Pr\left(\X>1\right)$
to denote $\L{\X}\left(\left\{ x:x>1\right\} \right)$. This is equal
to $\Pr\left(\left\{ \om\in\Om:\X\left(\om\right)>1\right\} \right)$
for any particular realization of $\X$ as a function on a probability
space $\left(\Om,\cA,\m\right)$.
\begin{example}
If $\X$ has the normal distribution with parameters $\mu$ and $\sigma$
then we say $\L{\X}=\N\left(\mu,\sigma\right)$; this distribution
is defined by $\L{\X}\left(B\right)=\frac{1}{\sigma\sqrt{2\pi}}\int_{B}e^{-\frac{\left(x-\mu\right)^{2}}{2\sigma^{2}}}\d x$
for Borel $B$.
\end{example}
If $\X$ is discrete, then $\L{\X}$ is just an assignment of a probability
to each possible value.
\begin{example}
If $\X$ is Poisson distributed with parameter $\lambda$, we write
$\L{\X}=\Po\left(\lambda\right)$; this is defined by $\Pr\left(\X=k\right)=\frac{\lambda^{k}e^{-\lambda}}{k!}$.\end{example}
\begin{defn}
The \emph{expected value }of a random variable $\X$ is $\E\X=\int x\d\L{\X}\left(x\right)$.
\end{defn}
For a random variable $\X$ that takes integer values, this definition
is equivalent to the well-known formula $\E\X=\sum_{x\in\ZZ}x\Pr\left(\X=x\right)$.

If we fix a particular underlying probability space $\left(\Om,\cA,\Pr\right)$,
we can also equivalently view expectation as a bounded linear functional
on the space of integrable functions: $\E\X=\int\X\left(\om\right)\d\Pr$.
So, $\E$ is defined in terms of a particular underlying probability
space. Sometimes we will define a new probability space $\left(\Om,\cA,\Pr'\right)$
by changing the measure on the same underlying set. In this case we
will write $\E_{\Pr'}$ to indicate expecation with respect to the
measure $\Pr'$, to avoid ambiguity.

In fact, the expectation functional defines its underlying probability
measure, because $\E\one_{\A}=\Pr\left(\A\right)$. Since the distribution
of a random variable is specified by a probability measure, the distribution
$\L{\X}$ of a random variable $\X$ also uniquely defines an expectation
functional $\E_{\X}:=\E_{\L{\X}}$.

\begin{comment}
\begin{defn}
The \emph{variance }of a random variable $\X$ is $\Var\X=\E\left(\X-\E X\right)^{2}$.
The \emph{$n$th moment} of a random variable $\X$ is $\E\left[\X^{n}\right]$.
\end{defn}
conditional variance?

\end{comment}
\begin{defn}
For two collections $S,S'\subseteq\cA_{1}$ of events, we say that
$S$ and $S'$ are \emph{independent} if $\Pr\left(\A\cap\A'\right)=\Pr\left(\A\right)\Pr\left(\A'\right)$
for each $\A\in S$ and $\A\in S'$. If $S=\left\{ \A\right\} $ contained
a single set, then we say $\A$ itself is independent of $S'$.
\end{defn}

\begin{defn}
Let $\left(\Om_{1},\cA_{1},\Pr\right)$ be a probability space and
$\left(\Om_{2},\cA_{2},\m\right)$ a measure space. Let $\X$ be a
random variable $\Om_{1}\to\Om_{2}$, and let $S$ be the set of all
events of the form $\left\{ \om\in\Om_{1}:\X\left(\om\right)\in\A_{2}\right\} $
for $\A_{2}\in\cA_{2}$. If $S$ is independent of $S'$ then we say
$\X$ itself is independent of $S'$.
\end{defn}
We can analogously say that two random variables are independent,
or a random variable and an event are independent, or any similar
combination.
\begin{defn}
If two objects are not independent, then we say they are \emph{dependent}.
\end{defn}

\begin{defn}
Suppose $\X:\Om_{1}\to\Om_{2}$ is a random element defined on these
spaces, and $\A_{1}\in\cA_{1}$ is an event with nonzero probability.
Then the \emph{distribution of $\X$ conditioned on $\A_{1}$} is
denoted by $\L{\X|\A_{1}}$ and defined by $\L{\X|\A_{1}}\left(\A_{2}\right)=\Pr\left(\X\in\A_{2}|\A_{1}\right)$
for $\A_{2}\in\cA_{2}$. The expected value of a random variable with
distribution $\L{\X|\A_{1}}$ is called the \emph{conditional expected
value of $\X$ given $\A_{1}$} and is denoted $\E\left[\X|\A_{1}\right]$.
\end{defn}
We can also define conditional expectation with respect to another
random variable. If $\X_{1}$ and $\X_{2}$ are random variables defined
on the same underlying probability space $\left(\Om,\cA,\Pr\right)$,
then the sets $\X_{2}^{-1}\left(B\right)$ for Borel $B$ comprise
a sub-$\sigma$-algebra $\cA'$ of $\cA$. Then, $\m:\A'\mapsto\E\left[\X_{1}\one_{\A'}\right]$
is a signed measure on $\cA'$ that is absolutely continuous with
respect to the restriction of $\Pr$ to $\cA'$. By the Radon-Nikodym
theorem there is an $\cA'$-measurable random variable $\E\left[\X_{1}|\X_{2}\right]$
that satisfies $\E\left[\X_{1}\one_{\A'}\right]=\E\left[\E\left[\X_{1}|\X_{2}\right]\one_{\A'}\right]$
for all $\A'$ in $\cA'$. This random variable is almost uniquely
defined: for any two choices of $\E\left[\X_{1}|\X_{2}\right]$, the
probability that they differ is zero.
\begin{defn}
The random variable $\E\left[\X_{1}|\X_{2}\right]$ as defined above
is called the \emph{conditional expectation of $\X_{1}$ with respect
to $\X_{2}$}. We can also view conditional expectation as a bounded
linear operator between functions: we define $\E^{\X_{2}}$ by $\X_{1}\mapsto\E\left[\X_{1}|\X_{2}\right]$.
\end{defn}
This definition generalizes the previous definition of expectation
conditioned on an event: if $\om\in\A$ and $\Pr\left(\A\right)>0$
then $\E\left[\X|\one_{\A}\right]\left(\om\right)=\E\left[\X|\A\right]$.

Note that if $\X_{2}$ is discrete then we do not need to invoke Radon-Nikodym.
We can define $\E\left[\X_{1}|\X_{2}\right]$ by $\E\left[\X_{1}|\X_{2}\right]\left(\om\right)=\E\left[\X_{1}|\X_{2}=\X_{2}\left(\om\right)\right]$
whenever $\Pr\left(\X_{2}=\X_{2}\left(\om\right)\right)>0$.


\subsection{Coupling}

Given a finite collection of measure spaces $\left(\Om_{1},\cA_{1},\m_{1}\right),\dots,\left(\Om_{n},\cA_{n},\m_{n}\right)$
recall the construction of the product measure space $\left(\Om,\cA,\m\right):=\left(\prod_{i=1}^{n}\Om_{i},\bigotimes_{i=1}^{n}\cA_{i},\prod_{i=1}^{n}\m_{i}\right)$.
If a random element takes values in a product space then each component
is measurable, and conversely if the components of a random tuple
are measurable then that tuple is measurable in the product space.
So, we can make the following definitions:
\begin{defn}
Given random elements $\X_{1},\dots,\X_{n}$ on the same underlying
probability space, $\L{\X_{1},\dots,\X_{n}}:=\L{\left(\X_{1},\dots,\X_{n}\right)}$
is called the \emph{joint distribution} of $\X_{1},\dots,\X_{n}$.
Conversely, given a random tuple $\left(\X_{1},\dots,\X_{n}\right)$,
each $\L{\X_{i}}$ is called a \emph{marginal distribution}.
\end{defn}
Suppose we have two distributions of random elements $\L{\X_{1}}$
and $\L{\X_{1}}$. \emph{Coupling} is the technique of constructing
a random ordered pair $\left(\X_{1},\X_{2}\right)$ which realizes
the given distributions as marginal distributions. Usually this is
done by specifying the joint distribution $\L{\X_{1},\X_{2}}$.

The idea is that coupling creates a particular kind of dependence
between $\X_{1}$ and $\X_{2}$ that allows us to compare the two
distributions. Often, we are able to make conclusions about the distributions
$\L{\X_{i}}$ which are independent of their specific realizations
as random elements in the coupling.


\subsection{Markov Chains}

\begin{note}

I'll need to define Markov Chains, stationary distributions and time-reversibility.

Perhaps I should talk more generally about stochastic processes, because
applying exchangeable pairs to Stein's method is has connections with
Ornstein-Uhlenbeck processes and also Stein's method can be applied
to Poisson processes.

\end{note}


\subsection{The Weak Topology on Probability Measures}

\global\long\def\cH{\mathcal{H}}


\global\long\def\h{h}


\global\long\def\TV{\mathrm{TV}}


\global\long\def\K{\mathrm{K}}


\global\long\def\W{\mathrm{W}}


\begin{note}

The main purpose of this section is to motivate the metrics usually
used in Stein's method: they are all legitimate topological metrics
and are consistent with the topology of convergence in distribution

\end{note}
\begin{defn}
Let $\left(\X_{n}\right)_{n\in\NN}$ be a sequence of random variables.
We say $\X_{n}$ \emph{converges in distribution} to a random variable
$\X$ if $\E f\left(\X_{n}\right)\to\E f\left(\X\right)$ for all
bounded continuous functions $f$. Alternatively, we say $\L{\X_{n}}$
converges \emph{weakly} to $\L{\X}$, or simply $\L{\X_{n}}\to\L{\X}$.
The topology on $\P\left(\RR\right)$ associated with this convergence
is called the \emph{weak topology} (we will see that it is indeed
a topology). Convergence in distribution of random vectors is defined
component-wise.
\end{defn}

\begin{defn}
The \emph{distribution function} $\F_{\X}$ of a random variable $\X$
is defined by $\F_{\X}\left(x\right)=\Pr\left(\X\le x\right)$.\end{defn}
\begin{thm}
\label{prop:dist}The following are equivalent.

\begin{enumerate}

\item \label{prop:dist-def}$\L{\X_{n}}\to\L{\X}$

\item \label{prop:dist-F}$\F_{\X_{n}}\left(x\right)\to\F_{\X}\left(x\right)$
for all $x$ where $\F_{\X}$ is continuous

\item (L\'evy's continuity theorem) $\E e^{it\X_{n}}\to\E e^{it\X}$
for all $t\in\RR$.

\end{enumerate}
\end{thm}
The equivalence of \ref{prop:dist-def,prop:dist-F} is a well-known
result called the Portmanteau Theorem.

When $\X$ and each $\X_{n}$ are integer random variables, then \ref{prop:dist-F}
reduces to the condition that $\Pr\left(\X_{n}=k\right)\to\Pr\left(\X=k\right)$
for all $k$. This characterization is usually used to prove the Poisson
limit theorem. L\'evy's continuity theorem is classically used to
prove the central limit theorem, but we will not discuss it in this
thesis.

For combinatorial applications, convergence in distribution can also
be proved by the ``method of moments'': if $\X$ is the only random
variable with the moments $\left(\E\X^{k}\right)_{k\in\NN}$, then
$\L{\X_{n}}\to\L{\X}$ if $\E\X_{n}^{k}\to\E\X^{k}$. Convergence
in distribution can also sometimes be inferred from stronger forms
of convergence when $\X$ and all the $\X_{n}$ are coupled to the
same underlying space.

A disadvantage of all these approaches is that it is difficult to
quantify the rate of convergence.

In functional analysis terms, note that expectation operators are
bounded linear functionals on the space of real bounded continuous
functions. Then, $\L{\X_{n}}\to\L{\X}$ just means that $\E_{\X_{n}}\to\E_{\X}$
in the weak-star topology. Although $C_{b}\left(\RR\right)^{*}$ is
not metrizable, the subspace corresponding to $\P\left(\RR\right)$
is in fact metrizable, with a metric called the L\'evy metric. For
Stein's method we will be interested in some slightly stronger metrics.
\begin{defn}
\label{def:general-metrics}For two probability measures $\Pr_{1},\Pr_{2}\in\P\left(\RR\right)$
and a collection of real measurable ``test'' functions $\cH$, define
$d_{\cH}$ by $d_{\cH}\left(\Pr_{1},\Pr_{2}\right)=\sup_{\h\in\cH}\left|\E_{\Pr_{1}}\h-\E_{\Pr_{2}}\h\right|$.
For random variables $\X_{1},\X_{2}$, we abuse notation to write
$d_{\cH}\left(\X_{1},\X_{2}\right)$ instead of $d_{\cH}\left(\L{\X_{1}},\L{\X_{2}}\right)$.
\end{defn}
Each $d_{\cH}$ is non-negative, symmetric and satisfies the triangle
inequality.
\begin{defn}
A set of real functions $\cH$ is a \emph{determining class} if $\E_{\Pr_{1}}\h=\E_{\Pr_{2}}\h$
for all $\h\in\cH$ implies that $\Pr_{1}=\Pr_{2}$.
\end{defn}
To check that $d_{\cH}$ is a metric, we only need to check that $d_{\cH}\left(\Pr_{1},\Pr_{2}\right)=0$
implies that $\Pr_{1}=\Pr_{2}$. That is, we need to check that $\cH$
is a determining class.
\begin{defn}
\label{def:special-metrics}We define some special cases of $d_{\cH}$.\begin{itemize}

\item If $\cH_{\K}=\left\{ \one_{\left(-\infty,x\right]}:x\in\RR\right\} $
then $d_{\K}:=d_{\cH_{\K}}$ is called the \emph{Kolmogorov metric}.

\item If $\cH_{\W}$ is the set of real functions $\h$ that satisfy
$\left|\h\left(x_{1}\right)-\h\left(x_{2}\right)\right|\le\left|x_{1}-x_{2}\right|$
for all $x_{1},x_{2}\in\RR$ (that is, the set of functions with Lipschitz
constant 1), then $d_{\W}:=d_{\cH_{\W}}$ is called the \emph{Wasserstein
metric}.

\item If $\cH_{\TV}$ is the set of functions $\one_{B}$ for Borel
$B$, $d_{\TV}:=d_{\cH_{\TV}}$ is called the \emph{total variation
metric}.\end{itemize}\end{defn}
\begin{prop}
\label{prop:metric}The Kolmogorov, Wasserstein and total variation
``metrics'' are actually metrics.\end{prop}
\begin{proof}
We check that $\cH_{\K}$, $\cH_{\W}$ and $\cH_{\TV}$ are determining
classes. Let $\cH\in\left\{ \cH_{\K},\cH_{\W},\cH_{\TV}\right\} $,
and suppose that $\E_{\Pr_{1}}\h=\E_{\Pr_{2}}\h$ for all $\h\in\cH$.
It suffices to prove that $\Pr_{1}\left(\left(-\infty,x\right]\right)=\Pr_{2}\left(\left(-\infty,x\right]\right)$
for all $x\in\RR$, since the sets $\left(-\infty,x\right]$ generate
the Borel $\sigma$-algebra. For $\cH\in\left\{ \cH_{\K},\cH_{\TV}\right\} $
this is immediate, because $\Pr_{\i}\left(\left(-\infty,x\right]\right)=\E_{\Pr_{\i}}\one_{\left(-\infty,x\right]}$.
So, consider, $\cH=\cH_{\W}$.

For $\varepsilon>0$ and $x\in\RR$, let $\h_{x,\varepsilon}$ be
the continuous function which takes the value 1 on the set $\left(-\infty,x\right]$,
takes the value 0 on the set $\left[x+\varepsilon,\infty\right)$,
and is linearly interpolated in the range $\left[x,x+\varepsilon\right]$.
Since $\varepsilon\h_{x,\varepsilon}\in\cH_{\W}$, we have $\E_{\Pr_{1}}\h_{x,\varepsilon}=\E_{\Pr_{2}}\h_{x,\varepsilon}$
for each $n\in\NN$. For each $x\in\RR$, $\h_{x,1/n}\to\one_{\left(-\infty,x\right]}$
pointwise and each $\h_{x,1/n}\le1$ so by the dominated convergence
theorem, $\E_{\Pr_{\i}}\h_{1/n}\to\E_{\Pr_{\i}}\one_{\left(-\infty,x\right]}$
for each $\i\in\left\{ 1,2\right\} $. We have again proved that $\Pr_{1}\left(\left(-\infty,x\right]\right)=\Pr_{2}\left(\left(-\infty,x\right]\right)$
for all $x\in\RR$.
\end{proof}

\begin{prop}
\label{prop:stronger-than-weak}The topologies induced by the Kolmogorov,
Wasserstein and total variation metrics are each stronger than the
weak topology.\end{prop}
\begin{proof}
If $d_{\K}\left(\X_{n},\X\right)\to0$ or $d_{\TV}\left(\X_{n},\X\right)\to0$
then $\F_{\X_{n}}\to\F_{\X}$ uniformly, so certainly \ref{prop:dist-F}
of \ref{prop:dist} holds.

Now, suppose $d_{\K}\left(\X_{n},\X\right)\to0$. Let $d_{n}=\sqrt{d_{\K}\left(\X_{n},\X\right)}$
and recall the definition of $\h_{x,\varepsilon}$ from the proof
of \ref{prop:metric}. Since $d_{n}\h_{x,d_{n}}\in\cH_{\K}$ for each
$n\in\NN$, we have $\E_{\X_{n}}\h_{x,d_{n}}-\E_{\X}\h_{x,d_{n}}\le d_{\K}\left(\X_{n},\X\right)/d_{n}=d_{n}\to0$
uniformly for $x\in\RR$. Now, note that $\F_{\X}\left(x-\varepsilon\right)\le\E_{\X}h_{x-\varepsilon,\varepsilon}\le\F_{\X}\left(x\right)\le\E_{\X}h_{x,\varepsilon}\le\F_{\X}\left(x+\varepsilon\right)$
for any random variable $\X$. If $\F_{\X}$ is continuous at $x$
then 
\begin{align*}
\F_{\X_{n}}\left(x\right)-\F_{\X}\left(x\right) & \le\left(\E_{\X_{n}}\h_{x,d_{n}}-\E_{\X}\h_{x,d_{n}}\right)+\left(\F_{\X}\left(x+d_{n}\right)-\F_{\X}\left(x\right)\right)\to0\\
\F_{\X_{n}}\left(x\right)-\F_{\X}\left(x\right) & \ge\left(\E_{\X_{n}}\h_{x-d_{n},d_{n}}-\E_{\X}\h_{x-d_{n},d_{n}}\right)+\left(\F_{\X}\left(x-d_{n}\right)-\F_{\X}\left(x\right)\right)\to0
\end{align*}
so \ref{prop:dist-F} of \ref{prop:dist} holds.
\end{proof}
\ref{prop:stronger-than-weak} tells us that we can sensibly use our
metrics to quantify the distance between random variables, in a way
that is consistent with distributional (weak) convergence. All three
metrics are relevant in their own right, but sometimes one may be
easier to work with. It is sometimes possible to transfer results
between metrics, though this usually results in worse constants than
working directly in the desired metric.

\begin{todo}It may be worthwhile to actually characterize the Wasserstein,
Kolmogorov and Total Variation topologies. In particular, Wikipedia
says that Wasserstein convergence is just weak convergence plus convergence
of the first moment.\end{todo}
\begin{defn}
If $\F_{\X}\left(x\right)=\int_{-\infty}^{x}f_{\X}\left(x\right)\d x$
then $f_{\X}$ is called the \emph{Lebesgue density} of $\X$, and
$\X$ is called a \emph{continuous} random variable.
\end{defn}
If $\X$ is a continuous random variable, then by the Radon-Nikodym
chain rule $\E_{\X}\h=\int_{\RR}\h\left(x\right)f_{\X}\left(x\right)\d x$.
\begin{prop}
Let $\X_{1},\X_{2}$ be random variables.\begin{enumerate}

\item \label{prop:transfer-K/TV}$d_{\K}\left(\X_{1},\X_{2}\right)\le d_{\TV}\left(\X_{1},\X_{2}\right)$

\item \label{prop:transfer-K/W}If $\X_{2}$ has Lebesgue density
bounded by $C$, then $d_{\K}\left(\X_{1},\X_{2}\right)\le\sqrt{2Cd_{\W}\left(\X_{1},\X_{2}\right)}$.

\end{enumerate}\end{prop}
\begin{proof}
(Adapted from \cite[Proposition 1.2]{Ros11}). \ref{prop:transfer-K/TV}
is immediate from the definition. Then, as in the proof of \ref{prop:stronger-than-weak},
\begin{eqnarray*}
\F_{\X_{n}}\left(x\right)-\F_{\X}\left(x\right) & \le & \left(\E_{\X_{n}}\h_{x,\varepsilon}-\E_{\X}\h_{x,\varepsilon}\right)+\left(\E_{\X}\h_{x,\varepsilon}-\F_{\X}\left(x\right)\right)\\
 & \le & d_{\W}\left(\X_{1},\X_{2}\right)/\varepsilon+\int_{x}^{x+\varepsilon}\h_{x,\varepsilon}f_{\X}\left(x\right)\d x\\
 & \le & d_{\W}\left(\X_{1},\X_{2}\right)/\varepsilon+C\varepsilon/2
\end{eqnarray*}
and similarly
\begin{eqnarray*}
\F_{\X_{n}}\left(x\right)-\F_{\X}\left(x\right) & \ge & -d_{\W}\left(\X_{1},\X_{2}\right)/\varepsilon-C\varepsilon/2,
\end{eqnarray*}
So, we can take $\varepsilon=\sqrt{2d_{\W}\left(\X_{1},\X_{2}\right)/C}$
to prove \ref{prop:transfer-K/W}.\end{proof}
\begin{example}
If $\L{\X_{2}}=\N\left(0,1\right)$ then $d_{\K}\le\left(2/\pi\right)^{1/4}\sqrt{d_{\W}\left(\X_{1},\X_{2}\right)}$.
\end{example}
In a combinatorial setting, many of our results are about integer
random variables. The total variation metric is usually exclusively
used in this case.
\begin{prop}
If $\X_{1},\X_{2}$ are integer-valued random variables, then
\[
d_{\TV}\left(\X_{1},\X_{2}\right)=\frac{1}{2}\sum_{k\in\ZZ}\left|\Pr\left(\X_{1}=k\right)-\Pr\left(\X_{2}=k\right)\right|.
\]
\end{prop}
\begin{proof}
For any Borel set $\A$, let $d_{\A}=\Pr\left(\X_{1}\in\A\right)-\Pr\left(\X_{2}\in\A\right)$,
so that $d_{\TV}\left(\X_{1},\X_{2}\right)=\sup\left|d_{\A}\right|$.
Define 
\begin{align*}
\A_{<} & =\left\{ k\in\ZZ:\,\Pr\left(\X_{1}=k\right)<\Pr\left(\X_{2}=k\right)\right\} ,\\
\A_{>} & =\left\{ k\in\ZZ:\,\Pr\left(\X_{1}=k\right)>\Pr\left(\X_{2}=k\right)\right\} .
\end{align*}
For any Borel $\A$, we have 
\begin{align*}
d_{\A} & =\sum_{k\in\ZZ\cap\A}\left(\Pr\left(\X_{1}=k\right)-\Pr\left(\X_{2}=k\right)\right)\\
 & \le\sum_{k\in\A_{>}}\left(\Pr\left(\X_{1}=k\right)-\Pr\left(\X_{2}=k\right)\right)\\
 & =d_{\A_{>}}
\end{align*}
and similarly $\Pr\left(\X_{1}\in\A\right)-\Pr\left(\X_{2}\in\A\right)\ge d_{\A_{<}}$.
Since $d_{\A_{>}}=-d_{\A_{<}}$, we have 
\[
d_{\TV}\left(\X_{1},\X_{2}\right)=\left(d_{\A_{>}}-d_{\A_{<}}\right)/2=\frac{1}{2}\sum_{k\in\ZZ}\left|\Pr\left(\X_{1}=k\right)-\Pr\left(\X_{2}=k\right)\right|.
\]

\end{proof}

\section{Random Combinatorial Structures\label{sec:random-structures}}

\global\long\def\G#1#2{\mathcal{G}_{#1,#2}}


\begin{note}

I'll leave this section until I'm sure what applications I'll be looking
at. Definitely random graph models, probably random permutations and
random matrices.

\end{note}


\section{Stein's Method in the Abstract}

\begin{note}There are a few quite different presentations of Stein's
method. One thing I'm trying to do here is to unify Stein's functional
analysis approach for exchangeable pairs \cite{Ste86} with Ross'
general presentation\cite{Ros11}.\end{note}

\global\long\def\cF{\mathcal{F}}


\global\long\def\T{T}


\global\long\def\U{U}


\global\long\def\cX{\mathcal{X}}


Suppose we have a potentially complicated random variable $\X$, and
we believe the distribution of $\X$ is close to a ``standard''
distribution $\L 0$. Then, Stein's method allows us to compare the
operators $\E_{\X}$ and $\E_{0}:=\E_{\L 0}$. This is sometimes directly
useful for approximating statistics of $\X$ (for example, $\Pr\left(\X\in\A\right)=\E_{\X}\one_{\A}$).
However, particularly for combinatorical applications, Stein's method
is most often used to bound the distance $d_{\cH}\left(\L{\X},\L 0\right)$,
where the metric $d_{\cH}$ from \ref{def:general-metrics} is defined
in terms of $\E_{\X}$ and $\E_{0}$.

Stein's method is motivated by the idea of a characterizing operator.
\begin{defn}
Let $\cF_{0}$ be a vector space and $\cX_{0}$ be a vector space
of measurable functions. We say a linear operator $\T_{0}:\cF_{0}\to\cX_{0}$
is a \emph{characterizing operator} for the distribution $\L 0$ if
$\im\T_{0}=\cX_{0}\cap\ker\E_{0}$. For convenience, where there is
no ambiguity we will often implicitly restrict $\E_{0}$ to $\cX_{0}$,
so we can write $\im\T_{0}=\ker\E_{0}$.
\end{defn}
The following proposition shows why $\T_{0}$ is called a characterizing
operator.
\begin{prop}
\label{prop:characterizing}If $\cX_{0}$ is a determining class and
$\im\T_{0}\subseteq\ker\E_{\X}$ then $\L{\X}=\L 0$.\end{prop}
\begin{proof}
If $\h\in\cX_{0}$, then $\h-\E_{0}\h\in\ker\E_{0}=\im\T_{0}$ so
$\E_{\X}\left[\h-\E_{0}\h\right]=0$. That is, $\E_{\X}\h=\E_{0}\h$
for all $\h\in\cX_{0}$, which means $\L{\X}=\L 0$ by the definition
of a determining class.\end{proof}
\begin{prop}
\label{prop:U_0-characterizing}$\T_{0}:\cF_{0}\to\cX_{0}$ is characterizing
if and only if there is a linear operator $\U_{0}:\cX_{0}\to\cF_{0}$
such that the following two equations hold.
\begin{align}
\E_{0}\T_{0} & =0_{\cF_{0}},\label{eq:E_0T_0-consistency}\\
\T_{0}\U_{0}+\E_{0} & =\id_{\cX_{0}}.\label{eq:U_0-characterizing}
\end{align}
\end{prop}
\begin{proof}
Suppose $\T_{0}$ is a characterizing operator. Equation \ref{eq:E_0T_0-consistency}
is immediate. Let $\left\{ \h_{\i}\right\} _{\i\in\mathcal{I}}$ be
a (Hamel) basis of $\cX_{0}$. For each $\i\in\mathcal{I}$ we have
$\h_{\i}-\E_{0}\h_{\i}\in\ker\E_{0}$ so there is some $f_{\i}$ (not
necessarily unique) that solves $\T_{0}f_{\i}=\h_{\i}-\E_{0}\h_{\i}$.
The operator $\U_{0}$ can then be defined by $\sum_{\i\in\mathcal{I}}a_{\i}\h_{\i}\mapsto\sum_{\i\in\mathcal{I}}a_{\i}f_{\i}$,
satisfying \ref{eq:U_0-characterizing}.

\begin{todo}there's probably a cleaner functional analysis way to
prove that. Also, is $\U_{0}$  bounded?\end{todo}

Conversely, suppose \ref{eq:E_0T_0-consistency} holds and $\U_{0}$
exists satisfying \ref{eq:U_0-characterizing}. For $\h\in\ker\E_{0}$
we have $\T_{0}\left(\U_{0}\h\right)=\h$ and $\h\in\im\T_{0}$, so
$\ker\E_{0}\subseteq\im\T_{0}$. Equation \ref{eq:E_0T_0-consistency}
immediately says that $\im\T_{0}\subseteq\ker\E_{0}$, so $\T_{0}$
is a characterizing operator.
\end{proof}
We'll use \ref{eq:U_0-characterizing} to give two important examples
of characterizing operators.
\begin{thm}
Define $T_{\N}$ by $T_{\N}f\left(x\right)=f'\left(x\right)-xf\left(x\right)$.
Let $\cX_{\N}$ be the set of integer-valued functions $\h$ that
satisfy $\E_{\N}\left|\h\right|<\infty$ and let $\cF_{\N}$ be the
set of integer-valued functions $f$ such that $\E_{\N}\left|\T_{\N}f\right|<\infty$.
Then $\T_{\N}:\cF_{\N}\to\cX_{\N}$ is a characterizing operator for
$\N\left(0,1\right)$.\end{thm}
\begin{proof}
For any $f\in\cF_{0}$, integration by parts gives
\[
\E_{\N}\T_{\N}f=\frac{1}{\sqrt{2\pi}}\int_{-\infty}^{\infty}e^{-t^{2}/2}f'\left(t\right)-\frac{1}{\sqrt{2\pi}}\int_{-\infty}^{\infty}te^{-t^{2}/2}f\left(t\right)\d t=0
\]
so $\E_{\N}\T_{\N}=0$ and \ref{eq:E_0T_0-consistency} holds. Then,
define $\U_{\N}$ by 
\[
\U_{\N}\h\left(x\right)=e^{x^{2}/2}\int_{-\infty}^{x}\left(\h\left(t\right)-\E_{\N}\h\right)e^{-t^{2}/2}\d t.
\]
By the product rule and the fundamental theorem of calculus, for all
$\h\in\cX_{\N}$ we have
\begin{align*}
\T_{\N}\U_{\N}\h\left(x\right) & =\h\left(x\right)-\E_{\N}\h,
\end{align*}
so \ref{eq:U_0-characterizing} holds and \ref{prop:U_0-characterizing}
completes the proof.\end{proof}
\begin{example}
Note that $\cH_{\K}\subseteq\cX_{\N}$, where $\cH$ is as defined
in \ref{def:special-metrics}
\end{example}

\begin{thm}
Define $\T_{\Po\left(\lambda\right)}$ by $\T_{\Po\left(\lambda\right)}f\left(k\right)=\lambda f\left(k+1\right)-kf\left(k\right)$.
Let $\cX_{\Po\left(\lambda\right)}$ be the set of integer-valued
functions $\h$ that satisfy $\E_{\Po\left(\lambda\right)}\left|\h\right|<\infty$
and let $\cF_{\Po\left(\lambda\right)}$ be the set of integer-valued
functions $f$ such that $\E_{\Po\left(\lambda\right)}\left|\T_{\Po\left(\lambda\right)}f\right|<\infty$.
Then $\T_{\Po\left(\lambda\right)}:\cF_{\Po\left(\lambda\right)}\to\cX_{\Po\left(\lambda\right)}$
is a characterizing operator for $\Po\left(\lambda\right)$.\end{thm}
\begin{proof}
For any $f\in\cF_{0}$, we have
\[
\E_{\Po\left(\lambda\right)}\T_{\Po\left(\lambda\right)}f=\sum_{\i=0}^{\infty}\frac{\lambda^{\i+1}}{\i!}f\left(\i+1\right)-\sum_{\i=1}^{\infty}\frac{\lambda^{\i}}{\left(\i-1\right)!}f\left(\i\right)=0
\]
so $\E_{\Po\left(\lambda\right)}\T_{\Po\left(\lambda\right)}=0$ and
\ref{eq:E_0T_0-consistency} holds. Then, define $\U_{\Po\left(\lambda\right)}$
by 
\[
\U_{\Po\left(\lambda\right)}\h\left(k\right)=\frac{\left(k-1\right)!}{\lambda^{k}}\sum_{\i=0}^{k-1}\frac{\lambda^{\i}}{\i!}\left(\h\left(\i\right)-\E_{\Po\left(\lambda\right)}\h\right).
\]
Substituting and simplifying gives
\[
\T_{\Po\left(\lambda\right)}\U_{\Po\left(\lambda\right)}\h\left(k\right)=\h\left(k\right)-\E_{\Po\left(\lambda\right)}\h,
\]
so \ref{eq:U_0-characterizing} holds and \ref{prop:U_0-characterizing}
completes the proof.
\end{proof}
\begin{todo} Stein chose $\cX_{0}=\left\{ \h:\E\left[\id_{\RR}^{k}\left|\h\right|\right]<\infty\mbox{ for all }k\right\} $,
for both the Poisson and Normal case. I'm not sure why, I'll revisit
this after looking at exchangeable pairs.\end{todo}

When we restrict our attention to integer-valued random variables,
$\cX_{\Po}$ is a determining class, because it contains every function
of the form $\one_{\A}$, where $\A$ is a set of integers. So, $\T_{\Po\left(\lambda\right)}$
is a characterizing operator in the sense of \ref{prop:characterizing}.


\subsection{Exchangeable Pairs}


\subsection{Size-Bias Coupling}


\part{Applications}

\begin{note}

I'd like to go into a number of small examples (perhaps interspersed
in the discussion of Stein's method in Part I), but I'd like to also
go through a number of ``big'' examples. I'd like these examples
to showcase
\begin{itemize}
\item different types of results: most applications give quantitative estimates.
\cite{Joh11} gives a non-quantitative distributional convergence
result that was not previously proved using other methods. There are
also results that have no connection with distribution metrics, such
as the concentration inequalities in \cite{Ros11}. In particular,
the Latin rectangle example in \cite{Ste86} is interesting in that
the final result is not probabilistic.
\item different types of distributions: definitely at least the Poisson
and normal case, perhaps also an example of a more exotic distribution
like the one in \cite{FS12} or perturbations of Poisson/normal distributions
as in \cite{BCX07}.
\item different ways to apply stein's method: definitely exchangeable pairs
and probably size-biasing. Maybe also Zero-bias coupling.
\end{itemize}
\end{note}

\bibliographystyle{amsalpha}
\bibliography{readings}

\end{document}
