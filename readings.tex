%% LyX 2.0.2 created this file.  For more info, see http://www.lyx.org/.
%% Do not edit unless you really know what you are doing.
\documentclass[11pt,english]{article}
\usepackage[T1]{fontenc}
\usepackage[latin9]{inputenc}
\usepackage{geometry}
\geometry{verbose,tmargin=3cm,bmargin=3cm,lmargin=3cm,rmargin=3cm,headheight=3cm,headsep=3cm,footskip=1.5cm}
\setlength{\parskip}{\bigskipamount}
\setlength{\parindent}{0pt}
\usepackage{amsthm}
\usepackage{amsmath}
\usepackage{setspace}
\setstretch{1.5}

\makeatletter
%%%%%%%%%%%%%%%%%%%%%%%%%%%%%% User specified LaTeX commands.
%\AtBeginDocument{
%\addtolength\abovedisplayskip{-10pt}
%\addtolength\belowdisplayskip{-5pt}
%}

\renewcommand\theenumi{\arabic{enumi}}
\renewcommand\labelenumi{(\theenumi)}

\let\originalleft\left
\let\originalright\right
\renewcommand{\left}{\mathopen{}\mathclose\bgroup\originalleft}
\renewcommand{\right}{\aftergroup\egroup\originalright}

\makeatother

\usepackage{babel}
\begin{document}

\title{Honours Readings Summaries}


\author{Matthew Kwan}

\maketitle
\global\long\def\floor#1{\left\lfloor #1\right\rfloor }


\global\long\def\ceil#1{\left\lceil #1\right\rceil }



\section{Combinatorial estimates by the switching method}

M. Hasheminezhad and B. D. McKay, Combinatorial estimates by the switching
method, Contemporary Mathematics, 531 (2010) 209-221.


\subsection{Summary}

Consider a finite set $\Omega$ of objects. A switching is a (nondeterministic)
operation that transforms one object into another (or more generally,
a switching is a relation $R\subseteq\Omega\times\Omega$). We can
partition $\Omega$ into subsets $\left\{ C\left(v\right)\right\} _{v\in V}$
and put a directed graph structure (possibly with loops) on the index
set $V$: if an element in $C\left(v\right)$ can switch to an element
in $C\left(w\right)$ then there is an arc between $v$ and $w$.

Hopefully, for each $v$ we can find a good lower bound $a\left(v\right)$
on the number of ways an $\omega\in C\left(v\right)$ can be switched,
and a good upper bound $b\left(v\right)$ on the number of switchings
an $\omega\in C\left(v\right)$ can be produced by. If we imagine
that all switchings are performed at once, $a$ and $b$ give bounds
on the inflow and outflow at each vertex in terms of the size of each
$C\left(v\right)$. (So, $a\left(v\right)$ and $b\left(v\right)$
can more generally be bounds on the \textit{average} number of switchings
per element in $C\left(v\right)$).

This gives some information about the relative sizes of the classes
$C\left(v\right)$. Given a set of vertices $X$, let $N\left(X\right)$
be the total amount of elements in all $C\left(x\right)$, where $x\in X$.
The objective of the paper is to bound $N\left(Y\right)/N\left(X\right)$
for given vertex sets $X$ and $Y$.

In order to obtain bounds on $N\left(Y\right)/N\left(X\right)$, we
write the constraints as a system of linear inequalities. Let $s'\left(v,w\right)$
be the (unknown) amount of switchings going from $C\left(v\right)$
to $C\left(w\right)$. For instance, for any vertex $v$, we have
$\sum s'\left(vw\right)\ge a\left(v\right)N\left(v\right)$. A nonzero
assignment of $s$ and $N$ satisfying the full set of inequalities
is called a \textit{feasible solution}, and a feasible solution which
maximizes $N\left(Y\right)/N\left(X\right)$ is called an \textit{optimal
solution}.

The key result of the paper is to show that optimal solutions always
take one of six standard forms. This is proved by a reduction of the
inequalities to a linear program, and the fact that an optimal solution
of a linear program always occurs at a vertex of the corresponding
convex polytope. The analysis is amenable to a general bound $\alpha$
on the arcs of the digraph instead of the functions $a,b$ (in that
case we have $\alpha\left(v,w\right)=b\left(w\right)/a\left(v\right)$).
Although$\alpha$ doesn't have a clear combinatorial interpretation,
the paper suggests that different $\alpha$ may arise from richer
information bounding the behaviour of the flow of switchings.

For a number of commonly-satisfied assumptions, the paper proves some
alternative bounds for $N\left(Y\right)/N\left(X\right)$ that are
slightly looser but easier to apply. In particular, a common use case
is that we have some statistic on the objects in $\Omega$ and we
believe that objects with a high value of that statistic make up a
negligible fraction of all the objects in $\Omega$. We would then
partition the objects according to our statistic, and design a switching
that tends to decrease the statistic, but by no more than some fixed
amount. We would choose $X$ to be the set of all vertices (partititions)
and $Y$ would be the set of all vertices (partitions) with a statistic
value higher than some $M$.


\subsection{Remarks}
\begin{itemize}
\item I'm not sure why $i_{k-1}=\max\left\{ M-\left(k-1\right)K,N+1\right\} $
is required for cases (a) and (b) of Lemma 3, instead of just $M-\left(k-1\right)K$
as in case (c).
\item In Corollary 1, it isn't spelled out what assumptions $X$ should
satisfy, but it would seem A1 still has to hold.
\item In Corollary 1, the somewhat inscrutable expression $k=\ceil{\frac{M+\min\left\{ 0,K-\rho-1\right\} }{K}}$
can be more easily seen to be $k=\ceil{\frac{M-\floor{\rho}}{K}}$.
\end{itemize}

\section{Asymptotic enumeration of sparse multigraphs with given degrees}

C. Greenhill and B. D. McKay, Asymptotic enumeration of sparse multigraphs
with given degrees, arXiv preprint arXiv:1303.4218 (2013)


\subsection{Summary}

The paper introduces a generalized version of the switching method,
where there are multiple different ``colours'' of switchings that
operate on the same set of objects. By slightly loosening the bound,
this generalized model can be reduced to an instance of the single-switching
model.

(unfinished)


\section{Asymptotic enumeration by degree sequence of graphs with degrees
$o\left(n^{1/2}\right)$}

B. D. McKay and N. C. Wormald, Asymptotic enumeration by degree sequence
of graphs with degrees $o\left(n^{1/2}\right)$, Combinatorica, 11
(1991) 369-382.


\subsection{Summary}

(unfinished)


\subsection{Remarks}
\end{document}
